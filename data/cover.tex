\thusetup{
  %******************************
  % 注意:
  %   1. 配置里面不要出现空行
  %   2. 不需要的配置信息可以删除
  %******************************
  %
  %=====
  % 秘级
  %=====
  secretlevel={秘密},
  secretyear={10},
  %
  %=========
  % 中文信息
  %=========
  ctitle={基于重写技术的\\ 嵌入式系统建模与验证},
  cdegree={工学博士},
  cdepartment={计算机科学与技术系},
  cmajor={计算机科学与技术},
  cauthor={刘嘉祥},
  csupervisor={顾 明教授},
  % cassosupervisor={孙家广教授}, % 副指导老师
  % ccosupervisor={某某某教授}, % 联合指导老师
  % 日期自动使用当前时间,若需指定按如下方式修改:
  % cdate={超新星纪元},
  %
  % 博士后专有部分
  cfirstdiscipline={计算机科学与技术},
  cseconddiscipline={系统结构},
  postdoctordate={2009年7月——2011年7月},
  id={编号}, % 可以留空: id={},
  udc={UDC}, % 可以留空
  catalognumber={分类号}, % 可以留空
  %
  %=========
  % 英文信息
  %=========
  etitle={Rewriting-Based Modeling and Verification of Embedded Systems},
  % 这块比较复杂,需要分情况讨论:
  % 1. 学术型硕士
  %    edegree:必须为Master of Arts或Master of Science(注意大小写)
  %             “哲学、文学、历史学、法学、教育学、艺术学门类,公共管理学科
  %              填写Master of Arts,其它填写Master of Science”
  %    emajor:“获得一级学科授权的学科填写一级学科名称,其它填写二级学科名称”
  % 2. 专业型硕士
  %    edegree:“填写专业学位英文名称全称”
  %    emajor:“工程硕士填写工程领域,其它专业学位不填写此项”
  % 3. 学术型博士
  %    edegree:Doctor of Philosophy(注意大小写)
  %    emajor:“获得一级学科授权的学科填写一级学科名称,其它填写二级学科名称”
  % 4. 专业型博士
  %    edegree:“填写专业学位英文名称全称”
  %    emajor:不填写此项
  edegree={Doctor of Philosophy}, 
  emajor={Computer Science and Technology},
  eauthor={Liu Jiaxiang},
  esupervisor={Professor Gu Ming},
  % eassosupervisor={Professor Sun Jiaguang},
  % 日期自动生成,若需指定按如下方式修改:
  % edate={December, 2005} 
  %
  % 关键词用“英文逗号”分割
  ckeywords={形式化方法, 嵌入式系统, 规范化条件重写模型, 建模, 验证},
  ekeywords={formal methods, embedded systems, normalization conditional rewrite systems, modeling, verification}
}

% 定义中英文摘要和关键字
\begin{cabstract}
随着嵌入式系统与现代社会生产、生活越来越深度的结合,其可靠性和安全性也变得与人们的生命财产安全息息相关,利用形式化方法对嵌入式系统进行验证以保证其正确性的需求日益迫切。形式化模型作为形式化验证方法的核心,是影响它在嵌入式系统中进行应用的关键因素。现代嵌入式系统中多线程技术的应用以及系统与外界环境复杂的交互,对形式化模型的建模能力和验证能力都提出了更高的要求。

重写模型适用于对多线程行为进行建模,且支持模型检测、定理证明等多种形式化验证技术,近年来受到形式化验证领域的关注。将重写模型应用于嵌入式系统的实际验证项目中,目前面临两个问题:(1) 如何在行为具有不确定性的重写模型中描述嵌入式软件的顺序行为;(2) 如何提高易用性,降低重写模型的建模成本。围绕这两个问题,本文提出了一个形式化模型——规范化条件重写模型,且基于该模型设计开发了一套针对嵌入式系统的建模方法以及一个针对 C 语言程序的终止性验证工具 \CTerm:

\begin{enumerate}
\item
针对重写模型对顺序行为表达能力的局限性,本文提出能够支持确定性行为描述的规范化条件重写模型。规范化条件重写模型支持自定义数据类型,具备描述状态等价、条件控制的表达能力,也能对以硬件并发行为为代表的不确定性行为、以及以软件顺序行为为代表的确定性行为进行描述和语义区分。
\item
针对模型易用性问题,以建模方法论为切入点,本文基于规范化条件重写模型,提出一套对嵌入式系统结构层次性、行为异构性、结构动态性和实时性等特征的建模方法,旨在对建模过程进行指导。基于语义映射的方式,本文对该建模框架予以实现,并通过对两个真实嵌入式系统进行应用,验证了该方法在嵌入式系统中实际应用的可行性。
\item 
针对模型易用性问题,以嵌入式系统软件的自动建模为另一切入点,本文开发了一套基于规范化条件重写模型的 C 程序终止性自动验证工具 \CTerm,为系统的完全正确性提供了必要的工具支持。
\end{enumerate}
\end{cabstract}

% 如果习惯关键字跟在摘要文字后面,可以用直接命令来设置,如下:
% \ckeywords{\TeX, \LaTeX, CJK, 模板, 论文}

\begin{eabstract}

Embedded systems are increasingly used in the modern society, becoming 
embedded in our lives. Due to the importance of their reliability and safety, 
formal methods for verifying the correctness of embedded systems are in great demand. As the heart of formal verification methods, formal models are the key to their application on embedded systems. Multithreading techniques are deployed in the modern systems, and there exists complicated interaction between embedded systems and their environment. Both will raise new challenges to the formal models about their abilities to model and verify.

Recently, rewrite systems have won attention for formal verification, given the facts that they are suitable for modeling the behaviors of multithreads and they support multiple verification techniques such as model checking and theorem proving. To apply rewrite systems on realistic verification projects for embedded systems, we need to answer two questions: (1) How do we capture sequential behaviors via rewrite systems inheriting non-determinism? (2) How do we enhance the usability of rewrite systems to ease the modeling process? To answer these questions, this dissertation introduces a new formal model named \emph{normalization conditional rewrite systems}. Based on the new model, we develop a methodology for modeling embedded systems and a termination prover \CTerm\xspace for C programs: 

\begin{enumerate} 
\item Aimed at the limitation of rewrite systems for expressing sequential behaviors, this dissertation introduces normalization conditional rewrite systems, which are able to describe deterministic behaviors. Normalization conditional rewrite systems support user-defined data types and the expressivity for state equivalence and conditional branching. They can also model and distinguish the semantics of non-deterministic behaviors, such as those concurrently in hardware, and of deterministic behaviors, such as those sequentially in software. 
\item Aimed at the usability of the model, focusing on modeling methodologies, this dissertation proposes a methodology which captures the hierarchy, the heterogeneous behaviors, the dynamic structures and the real-timeness of embedded systems. We implement the methodology based on semantics mapping. Then we demonstrate its feasibility for application, by applying it on two realistic embedded systems. 
\item Aimed at the usability of the model, focusing on the automatic construction of models of embedded software, this dissertation develops an automatic termination prover \CTerm\xspace for C programs based on normalization conditional rewrite systems. The tool provides the possibility to prove the total correctness of systems. 
\end{enumerate}

\end{eabstract}

% \ekeywords{\TeX, \LaTeX, CJK, template, thesis}
