\chapter{绪论}
\label{cha:intro}

\section{研究背景}


嵌入式系统最早出现在六十年代,其产生是为了减少搭载系统的体积重量以及降低成本。由于作为计算资源的成本高昂的计算机逐步被成本相对低廉的微处理器系统替换,设备不仅变得小型化、价格降低,同时处理能力以及功能也不断提高。这些变化使得七八十年代时期,民用电子、消费类电子等行业得到了快速发展并大规模兴起。从工业 2.0 时代开始的自动化、电力驱动的大规模工业生产中,早期的嵌入式系统开始大量投入使用。到现在的工业3.0时代以及2013年德国提出的工业4.0,信息化、智能化、自动化、网络化作为核心,数字化产品以及产品生产制造过程,其全生命周期都将依赖于电子信息而产生的新工业模式。在这中间,嵌入式系统的不断强大将为工业的进程提供技术支撑,以加快工业4.0在全领域的普及。

最早,不同的嵌入式系统承载不同功能,其设计是为特定系统、特定任务而定制的,大多数嵌入式系统都是功能单一的。在发展过程中,嵌入式系统的处理单元逐渐变得强大(如现在的ARM、PowerPC、MIPS等),外围设备、硬件资源越来越丰富,这使其可以承担更多复杂的处理任务,并具有了通用性、架构可移植的特点。在嵌入式系统硬件系统能力提升的基础上,除了为硬件系统开发固件外,Linux、MSDOS,以及专为嵌入式系统开发的实时操作系统VxWorks等操作系统的加入,可以帮助系统中设备的协调调度、资源监管等,并且使得开发运行针对应用需求的用户控制界面或更多功能的应用软件成为可能。 

除消费类电子产品外,嵌入式系统在民用、军用大型设备中的作用同样重要。在航空领域中,民用航空飞机的机载电子设备功能愈加强大,系统架构愈加复杂。飞行控制系统、导航系统、通信系统、雷达系统、环境控制系统,以及各终端传感器系统都搭载了嵌入式系统来对其进行控制,并完成计算、存储、传输等以支持飞控、通信、导航等高级任务。其中系统异步、并发、调度、实时等问题引起的不确定性都是影响系统正常工作、影响飞行安全的重要因素。安全性是民用飞机的重要属性,系统安全性的设计、验证以及管理贯穿飞行器生命周期。有 ARP4754~\cite{ARP4754A}、ARP4761~\cite{ARP4761} 来指导飞行器系统的安全性设计,以及 DO-178、DO-254 来指导符合安全性等级的软件、硬件设计。其中嵌入式系统的硬件要满足安全性设计中的可靠性需求。而嵌入式系统中的固件及软件系统要满足软件的验证需求。如飞控系统的安全性等级为A级,其涉及的嵌入式系统硬件设备要满足安全性保障等级为A的设计、测试、验证流程。对于嵌入式系统软件,若其应满足的安全性保证等级为A,则要求其设计及验证过程中必须使用严格的形式化方法对其安全等级进行保障。其它安全攸关的系统,如轨道运输、海洋运输都有相应的安全性保障需求。嵌入式系统硬件设计验证方法早已比较成熟,但是软件的形式化设计验证方法相对于硬件起步较晚。随着嵌入式系统软件发展的加快,针对实时性嵌入式系统软件的形式化设计与验证有迫切需求。在嵌入式系统的设计与开发过程中,硬件与软件往往是难以分开单独设计的,因为其软件及固件开发的专用性,互相影响程度很深。所以一种基于形式化方法的、可以同时对硬件及软件系统进行验证的方法亟待研究。



\section{研究现状}

对嵌入式系统的建模方法不胜枚举,根据其描述对象的不同,可以建立架构模型、状态模型、数据模型等。它们分别以不同角度去描述嵌入式系统,用仿真、测试、形式化验证等方法保证系统的正确性、有效性、可靠性等多方面重要属性,特别是对安全攸关系统的设计、验证与运行提供有力支撑。如上一节提到,嵌入式系统在众多安全攸关系统(如航空航天飞行器、铁路、核电站、医疗、保密系统等)中扮演着重要的角色。自1990年起,已有多份统计以研究报告与论文的形式指出形式化方法在工业领域中所取得的突出成果,2009年英国约克大学的Jim Woodcock统计得到:在工业应用中,对比只依赖于测试验证的工程开发,基于形式化建模的工程项目有 92\% 得到了产品品质的提升,体现在更少的故障、正确性的提升、设计能力提升等多方面。

在这些形式化方法中,有些用抽象的形式化规约避免模型中的二义性,从而保证设计质量,如VDM~\cite{DBLP:conf/fm/1978}、Z~\cite{DBLP:books/daglib/0067866} 等。有些用状态模型描述系统行为特性,模型本身能够采用静态分析与模型执行来进行验证,如Petri网~\cite{murata1989}、自动机~\cite{DBLP:books/aw/HopcroftU79} 等。 

有限状态机~\cite{minsky1967computation}(FSM)是面向系统状态最基本的模型,被控制系统广泛地应用。但对于功能、架构复杂的嵌入式系统,有限状态机缺乏对其并发性、实时性等方面的支持。所以有限状态机有多种扩展形式:支持并发性的层次化并发有限状态机~\cite{DBLP:journals/scp/Harel87}(Hierarchical Concurrent FSM,HCFSM)以及支持时间的时间自动机~\cite{DBLP:journals/tcs/AlurD94,DBLP:conf/cav/Alur99}(Timed automata)。HCFSM可将一个状态组视为一个状态,状态组之间通过全局变量进行通信,可以用于描述相对复杂的控制系统。时间自动机作为有限状态机的另一个变种,增加对状态转移过程中时间因素的描述,变迁上标记的是该状态转移发生的时间约束,从而可以满足对实时系统时间分析的需求。

{\sc Uppaal2k}~\cite{larsen1999uppaal2k},作为 {\sc Uppaal}~\cite{DBLP:journals/sttt/LarsenPY97} 的后继工具,提供了对时间系统的建模、仿真以及验证功能。适用于对具有共享变量、通道通信的系统进行描述。UppAal2k 支持对时间自动机图形化的描述、动态仿真以及模型检测等功能,可以进行有界活性检测、死锁检测、验证使用TCTL(Timed Computation Tree Logic)表达式描述的多种性质。

Petri网的研究与应用相当广泛。由于对异步并发系统描述上的优势,Petri 网在实时系统、协议验证、硬件设计、制造过程、商业管理等众多领域中均发挥了作用。在实时嵌入式系统的应用中,Petri 网为描述待验证系统的异步、并发性、不确定性等多种特性提供了有效工具。为了扩展标准Petri网的适应性,Petri 网衍生出多种变种,如着色 Petri 网(Colored Petri net,CPN)、混合Petri网(Hybrid Petri net)、随机Petri网(Stochastic Petri net),以及多种时间相关的Petri网,如Time Petri net(TPN)、Timed Petri net(TdPN)、Timing Constraint Petri net(TCPN),以适用于不同的应用需求。

在以嵌入式实时系统为验证对象时,时间是最重要的因素之一,时间自动机以及各类时间Petri网均可在一定程度上满足描述以时间为关键因素的实时系统。根据描述对象的需求,将时间因素加入Petri网三个要素——库所、变迁以及有向弧上,扩展成为各种时间Petri网。这些时间Petri网虽然在描述时间约束上有不同的设计,但由于其不脱离标准Petri 网的形式化语法语义,基本可以实现相互之间的转化。TdPN以在变迁上的变迁时刻与延迟时间来表示该变迁转换时允许的最迟时间,即在规定阈值内到位的令牌仅可供该变迁使用。区别于TdPN,TPN所描述的系统不但要满足变迁上相应的时间延迟上限,而且需要满足变迁发生前所花费的最小时间,将变迁发生时间看作一个时间区间,以便更精确地对系统的运行时间进行分析与评估。由于其特点,TdPN与TPN都对并发系统具有一定的分析评估能力。而TCPN在库所与变迁上均可以增加时间约束,以表示触发过程的时间。与TdPN以及TPN不同,TCPN采用弱触发规则,即不满足时间约束时,由调度决策决定是否触发该变迁。因此,TCPN可以更准确地描述系统的执行过程。

在工具方面,一共有多达40多种工具提供了对时间相关的Petri网的支持。TINA(Time petri Net Analyzer)是其中应用最广泛的一种,可以对TPN和TdPN进行线性时序逻辑(Linear Temporal Logic,LTL)性质验证、可达性验证等。Romeo是支持TPN验证的工具,并且预定义了多种从 TPN 到时间自动机的转换方法。这些工具基本都支持Petri网的建模、仿真及多种验证工作。

对于前面提到的时间自动机以及几种时间Petri网,由于它们的语法、以及其工具的局限性,不支持对系统中复杂的数据类型进行建模。由于语言表达能力不足,使得这些建模方法在对涉及复杂数据结构(如结构式数据类型或自定义数据类型)的系统的描述能力具有局限性。

在Petri网的众多变种中,还有一个应用广泛的变种名为着色Petri网(Colored Petri net),其具备更强的表达能力。着色 Petri 网可以对自身的令牌(token)对象进行具体的特征描述,即着色。并且它支持利用面向表达式的函数式程序设计语言对系统进行描述,可以支持如布尔、整数、列表、字符、结构体等多种数据类型,丰富了表达能力。通过CPN tool可以对使用CPN的项目进行建模、仿真、验证, 同时CPN tool具有对函数式程序设计语言标准 ML(SML)的支持。有了以上特点,CPN 可以用来对具有复杂表达对象的系统进行建模及验证工作。

重写模型是完全不同于自动机的数学模型,它是基于代数表达式和规则的一种不确定性模型。近年来重写模型开始被应用于系统的建模与验证工作中。 Bluespec是一种基于重写模型的硬件描述语言,十分适用于SoCs(System on Chips)的设计验证。Bluespec 使用重写的目的是利用重写模型本身并发的特点来描述并发系统,以便通过仿真来保障其设计正确性。它利用重写规则来描述并发,可以降低设计的复杂性。Bluespec支持函数式程序设计语言Haskell,以便对嵌入式系统中的复杂类型、自定义参数等进行定义,增加其描述能力。其工具Bluespec System Verilog可以将Bluespec写成的代码转换成RTL(Resistor Transistor Logic)。但工具本身没有支持形式化验证的工具,需要通过外部工具进行。重写逻辑(Rewriting logic)是重写模型的一种扩展,它通过对重写规则以及规则应用策略的扩展,增加其对复杂数据结构及复杂系统结构的描述能力。基于重写逻辑的 Maude 语言及工具,可支持利用重写逻辑对系统进行建模,并支持模型仿真、可达性验证、线性时序逻辑性质验证等功能。目前基于重写逻辑的建模验证工作主要针对协议验证及语言验证,还没有被广泛应用于嵌入式系统领域。





\section{研究思路}
%\label{chap1:sample:table}


\section{论文贡献}
\label{sec:theorem}

\section{论文结构}
\label{sec:bib}
