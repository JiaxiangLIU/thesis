\chapter{C 语言程序终止性的自动验证}

本章探讨规范化条件重写模型在建模验证中的另一个应用:对 C 语言程序的终止性进行自动验证。

\section{引言}

\section{C 程序的整数重写模型}

对 C 程序的终止性进行自动验证,其关键在于自动化构建 C 程序对应的整数重写模型。本文采取文献~\inlinecite{DBLP:conf/cade/StroderGBFFHS14} 的方法,基于符号执行技术构建 C 程序的符号执行图,再从符号执行图生成适用于终止性验证的整数重写模型。

\begin{figure}[ht]
\centering
\includegraphics[width=\textwidth]{C2intTRS.jpg}
\caption{C 程序的整数重写模型自动构造}
\label{f:C2intTRS}
\end{figure}

图~\ref{f:C2intTRS} 是 C 程序的整数重写模型自动构造流程。C 程序经过编译器 Clang 的编译后生成对应的 LLVM IR 代码。基于这些 LLVM IR 代码,利用符号执行技术,产生输入程序的符号执行图。整数变迁系统描述了该符号执行图的终止性行为。最后利用整数重写模型丰富的表达能力,将该整数变迁系统的语义编码成重写模型的形式。

这个构造过程涉及到三个重要的模型:符号执行图、整数变迁系统和整数重写模型。在本小节接下来的部分将对这三个概念进行介绍,并对其构造的核心原理进行描述。

\subsection{符号执行图}
\label{ss:seg}

\emph{符号执行图}(Symbolic execution graph)是由 Giesl 等人提出的用于描述程序执行过程的数据结构~\cite{DBLP:conf/lopstr/GieslSSEF12}。一个符号执行图是一个有向图,它包含一个顶点集合和一个边集合。每个顶点代表一个\emph{抽象}的程序执行状态,它包含程序当时的位置(pc)、变量赋值状态、内存分配情况等信息。程序执行图的边分为三类:
\begin{enumerate} [(i)]
\item 求值(Evaluation)类型。求值类型的边表示:若其起始顶点表示的程序状态为 $s$,当前程序位置为 $p$($p$ 也是状态 $s$ 中包含的信息),则 $s$ 在执行 $p$ 位置的程序指令后,其状态 $s'$ 为该有向边的终止顶点代表的状态。
\item 精化(Refinement)类型。精化类型的边的终止顶点 $v$ 所表示的状态是起始顶点 $u$ 表示的状态的精化,即 $v$ 所表示的程序状态集合是 $u$ 所表示的程序状态集合的子集。
\item 抽象(Abstraction)类型。抽象类型的边与精化类型的边相反,它表示其终止顶点 $v$ 的状态是起始顶点 $u$ 状态的抽象,即 $u$ 所表示的程序状态集合是 $v$ 所表示的程序状态集合的子集。
\end{enumerate}

至于每个顶点所表示的抽象程序状态,本文基于文献~\inlinecite{DBLP:conf/cade/StroderGBFFHS14} 的定义,进行了简化:
\begin{definition}[抽象程序状态] 
\label{d:abstract-state}
给定程序变量集合 $\VP$,符号化变量集合 $\Vsym$,程序位置集合 $Pos$,则一个抽象程序状态是一个六元组 $\lb p, LV, LAL, KB, AL, PT \rb$。其中 $p\in Pos$;$LV : \VP \ra \Vsym$;$LAL = \{\interp{v_1,v_2}\mid v_1,v_2\in\Vsym, v_1\le v_2 \}$;$KB\subseteq QF\_IA(\Vsym)$;$AL = \{\interp{v_1,v_2}\mid v_1,v_2\in\Vsym, v_1\le v_2 \}$;$PT \subseteq \{(v_1\pta_{\ty} v_2)\mid v_1,v_2\in\Vsym, \ty \mbox{是LLVM的类型}\}$。另外,抽象程序状态 $ERR$ 表示可能违背了内存安全的状态;抽象程序状态 $END$ 表示执行结束的程序状态。
\end{definition}

直观上解释,$LV$(Local variables)表示该状态的程序变量赋值情况,程序变量值由集合 $\Vsym$ 中的符号化变量表示;$LAL$(Local allocation list)表示局部内存分配情况,二元组 $\interp{v_1,v_2}$ 表示在 $v_1$ 和 $v_2$ 之间的内存单元是已经过分配的;$QF\_IA(\Vsym)$ 是指\emph{无量词的}(Quantifier-free)一阶公式,用于描述 $\Vsym$ 变量的\emph{整数代数}(Integer arithmetic)性质,$KB$(Knowledge base)是该状态满足的性质集合;$AL$(Allocation list)与 $LAL$ 类似,表示的是全局内存分配情况;$PT$(Pointer table)表示内存状态,元组 $(v_1\pta v_2)$ 表示内存地址 $v_1$ 指向的内容是 $v_2$,具有类型 $\ty$。

定义~\ref{d:abstract-state} 与文献~\inlinecite{DBLP:conf/cade/StroderGBFFHS14} 对抽象程序状态定义的最主要区别是,定义~\ref{d:abstract-state} 描述了单函数的抽象状态,而文献~\inlinecite{DBLP:conf/cade/StroderGBFFHS14} 的定义包含了栈结构,允许描述多函数之间的调用。

基于我们的简化定义,我们定义用于描述每个抽象程序状态的一阶谓词集合 $\lb a\rb$:
\begin{definition}
给定抽象程序状态 $a$,一阶谓词集合 $\lb a\rb$ 是满足以下条件的最小集合:
\begin{eqnarray}
\lb a \rb & = & KB 
\cup \{1\le v_1\land v_1\le v_2 
    \mid \interp{v_1, v_2} \in LAL \cup AL  \} \nonumber \\
& & \cup\; \{ v_2 < w_1 \lor w_2 < v_1 
    \mid \interp{v_1,v_2},\interp{w_1,w_2}\in LAL\cup AL, (v_1,v_2)\not= (w_1,w_2) \}    \nonumber \\
& & \cup\; \{ v_2 = w_2 \mid (v_1\pta_{\ty}v_2),(w_1\pta_{\ty}w_2)\in PT 
    \mbox{ and } \vDash \lb a\rb \Dra  v_1=w_1   \} \nonumber \\
& & \cup\; \{ v_1 \not= w_1 \mid (v_1\pta_{\ty}v_2),(w_1\pta_{\ty}w_2)\in PT 
    \mbox{ and } \vDash \lb a\rb \Dra  v_2\not=w_2   \} \nonumber \\
& & \cup\; \{ v_1 > 0 \mid (v_1\pta_{\ty} v_2) \in PT \} \;\;\mbox{。} \nonumber
\end{eqnarray}
\end{definition}

需要注意,$\lb a\rb$ 是归纳定义的。$\lb a \rb$ 定义了抽象程序状态 $a$ 满足的一阶谓词性质集合。

符号执行图的构建基于符号执行规则进行。给定一个抽象程序状态,根据符号执行规则计算下一个(或两个)抽象程序状态。由于实际的程序状态空间可能是无穷的,为了用符号执行图的有穷顶点数表示无穷的程序状态空间,在构造过程中需要根据一定的策略对抽象程序状态进行抽象。下面先介绍符号执行规则,再介绍符号执行图的构造策略。

基于定义~\ref{d:abstract-state},我们对文献~\inlinecite{DBLP:conf/cade/StroderGBFFHS14} 和~\inlinecite{DBLP:journals/jar/StroderGBFFHSA17} 的符号执行规则进行简化。以下列举其中较关键的几条符号执行规则。


\begin{table}[htbp]
\caption{符号执行规则:\texttt{load} 指令(内存已分配)}
\label{tab:rule-load-alloc}
\begin{tabularx}{\textwidth}{|X|}
\hline
\textbf{\texttt{load}} \emph{已分配内存} \\
{\centering $
\inferrule
   {\lb p, LV, LAL, KB, AL, PT\rb}
   {\lb p^+, LV[\texttt{x}:= w], LAL, KB, AL, PT\cup\{LV(\texttt{ad})\pta_{\ty} w \}\rb }
$ \\}
\textbf{如果满足以下条件} \\
~$\bullet$ $p$ : "\texttt{x = load ty* ad}" 其中 $\texttt{x},\texttt{ad}\in\VP$; \\
~$\bullet$ 存在 $\interp{v_1,v_2}\in LAL\cup AL$ \newline 
~\phantom{$\bullet$ } 使得 $\vDash \lb a\rb \Dra (v_1\le LV(\texttt{ad}) \land LV(\texttt{ad}) + size(\ty)-1\le v_2)$;  \\
~$\bullet$ $w\in\Vsym$ 是新变量。 \\
\hline
\end{tabularx}
\end{table}

\begin{table}[htbp]
\caption{符号执行规则:\texttt{load} 指令(内存未分配)}
\label{tab:rule-load-unalloc}
\begin{tabularx}{\textwidth}{|X|}
\hline
\textbf{\texttt{load}} \emph{未分配内存} \\
{\centering $
\inferrule
   {\lb p, LV, LAL, KB, AL, PT\rb}
   {ERR}
$ \\}
\textbf{如果满足以下条件} \\
~$\bullet$ $p$ : "\texttt{x = load ty* ad}" 其中 $\texttt{x},\texttt{ad}\in\VP$; \\
~$\bullet$ 不存在 $\interp{v_1,v_2}\in LAL\cup AL$ \newline 
~\phantom{$\bullet$ } 使得 $\vDash \lb a\rb \Dra (v_1\le LV(\texttt{ad}) \land LV(\texttt{ad}) + size(\ty)-1\le v_2)$。  \\
\hline
\end{tabularx}
\end{table}

表~\ref{tab:rule-load-alloc} 和~\ref{tab:rule-load-unalloc} 分别是 \verb|load| 指令从已分配内存和未分配内存中读取数据时对应的符号执行规则。如表~\ref{tab:rule-load-alloc} 所示,当 \verb|load| 指令准备从地址 \verb|ad| 中读取数据时,需要先判断是否能根据当前抽象程序状态 $a$ 的谓词表达式 $\lb a \rb$ 推断出地址 \verb|ad| 对应的内存块 $\interp{LV(\texttt{ad}), LV(\texttt{ad})+ size(\ty) -1}$ 已经经过分配。如果该内存块已经经过分配,则可执行表~\ref{tab:rule-load-alloc} 中的符号执行规则生成新的抽象程序状态,新状态中的程序变量 \verb|x| 得到更新,程序指针指向 $p$ 的下一条指令 $p^+$。如果不能推断出该内存块已经经过分配,则如表~\ref{tab:rule-load-unalloc} 所示,将产生抽象程序状态 $ERR$。

\begin{table}[htbp]
\caption{符号执行规则:\texttt{icmp eq} 指令(命题成立)}
\label{tab:rule-icmp-eq-true}
\begin{tabularx}{\textwidth}{|X|}
\hline
\textbf{\texttt{icmp eq}} \emph{命题为真} \\
{\centering $
\inferrule
   {\lb p, LV, LAL, KB, AL, PT\rb}
   {\lb p^+, LV[\texttt{x} := w], LAL, KB\cup\{w=1\}, AL, PT\rb}
$ \\}
\textbf{如果满足以下条件} \\
~$\bullet$ $p$ : "\texttt{x = icmp eq ty $t_1$, $t_2$}" 其中 $\texttt{x}\in\VP$ 且 $t_1,t_2\in\VP\cup\mathbb{Z}$; \\
~$\bullet$ $\vDash \lb a\rb \Dra (LV(t_1) = LV(t_2))$;  \\
~$\bullet$ $w\in\Vsym$ 是新变量。 \\
\hline
\end{tabularx}
\end{table}

\begin{table}[htbp]
\caption{符号执行规则:\texttt{icmp eq} 指令(命题不成立)}
\label{tab:rule-icmp-eq-false}
\begin{tabularx}{\textwidth}{|X|}
\hline
\textbf{\texttt{icmp eq}} \emph{命题为假} \\
{\centering $
\inferrule
   {\lb p, LV, LAL, KB, AL, PT\rb}
   {\lb p^+, LV[\texttt{x} := w], LAL, KB\cup\{w=0\}, AL, PT\rb}
$ \\}
\textbf{如果满足以下条件} \\
~$\bullet$ $p$ : "\texttt{x = icmp eq ty $t_1$, $t_2$}" 其中 $\texttt{x}\in\VP$ 且 $t_1,t_2\in\VP\cup\mathbb{Z}$; \\
~$\bullet$ $\vDash \lb a\rb \Dra (LV(t_1) \not= LV(t_2))$;  \\
~$\bullet$ $w\in\Vsym$ 是新变量。 \\
\hline
\end{tabularx}
\end{table}

表~\ref{tab:rule-icmp-eq-true} 和~\ref{tab:rule-icmp-eq-false} 是指令 \verb|icmp| 的符号执行规则。这里只列出了 \verb|icmp eq| 的具体规则,其它比较操作(如 \verb|ule|、\verb|uge| 等)的规则类似。如果当前状态的性质集合 $\lb a\rb$ 可以推断出 \verb|icmp eq| 指令对应命题为真,如表~\ref{tab:rule-icmp-eq-true} 所示,则下一状态将程序变量 \verb|x| 的值赋为 $w$,$KB$ 告诉我们 $w=1$。否则,如果 $\lb a\rb$ 足以推断出该命题为假,则 \verb|x| 对应的值为 0。
如果 $\lb a\rb$ 所包含的信息不能推断出目标命题的真假,这说明下一个状态需要分情况讨论。于是我们需要往当前的抽象程序状态加入更多的信息,对状态进行“精化”。

\begin{table}[htbp]
\caption{符号执行规则:精化(\texttt{icmp eq} 指令)}
\label{tab:rule-refine-icmp-eq}
\begin{tabularx}{\textwidth}{|X|}
\hline
\emph{对} \textbf{\texttt{icmp eq}} \emph{进行精化}  \\
{\centering $
\inferrule
   {\lb p, LV, LAL, KB, AL, PT\rb}
   {\lb p, LV, LAL, KB\cup\{\varphi\}, AL, PT\rb \;\;\mid\;\; \lb p, LV, LAL, KB\cup\{\lnot\varphi\}, AL, PT\rb}
$ \\}
\textbf{如果满足以下条件} \\
~$\bullet$ $p$ : "\texttt{x = icmp eq ty $t_1$, $t_2$}" 其中 $\texttt{x}\in\VP$ 且 $t_1,t_2\in\VP\cup\mathbb{Z}$; \\
~$\bullet$ $\not\vDash \lb a\rb \Dra \varphi$ 且 $\not\vDash \lb a\rb \Dra \lnot\varphi$; \\
~$\bullet$ $\varphi$ 是 $LV(t_1) \not= LV(t_2)$。 \\
\hline
\end{tabularx}
\end{table}

表~\ref{tab:rule-refine-icmp-eq} 是对指令 \verb|icmp eq| 进行精化的规则。若根据当前抽象状态的信息不能推断出命题 $\varphi$ 的真假,则精化规则产生两个新的抽象程序状态,两个状态的集合 $KB$ 分别加入了 $\varphi$ 的真假信息,使后续的状态求值可以继续进行。其它涉及条件比较的指令的精化规则,也可以类似地进行定义。

\begin{table}[htbp]
\caption{符号执行规则:\texttt{add} 指令}
\label{tab:rule-add}
\begin{tabularx}{\textwidth}{|X|}
\hline
\textbf{\texttt{add}} \\
{\centering $
\inferrule
   {\lb p, LV, LAL, KB, AL, PT\rb}
   {\lb p^+, LV[\texttt{x} := w], LAL, KB\cup\{w= LV(t_1) + LV(t_2)\}, AL, PT\rb} 
$ \\}
\textbf{如果满足以下条件} \\
~$\bullet$ $p$ : "\texttt{x = add ty $t_1$, $t_2$}" 其中 $\texttt{x}\in\VP$ 且 $t_1,t_2\in\VP\cup\mathbb{Z}$; \\
~$\bullet$ $w\in\Vsym$ 是新变量。 \\
\hline
\end{tabularx}
\end{table}

表~\ref{tab:rule-add} 展示了代数运算指令 \verb|add| 的符号执行规则。

\begin{table}[htbp]
\caption{符号执行规则:\texttt{alloca} 指令(分配错误)}
\label{tab:rule-alloca-err}
\begin{tabularx}{\textwidth}{|X|}
\hline
\textbf{\texttt{alloca}} \emph{失败} \\
{\centering $
\inferrule
   {\lb p, LV, LAL, KB, AL, PT\rb}
   {ERR} 
$ \\}
\textbf{如果满足以下条件} \\
~$\bullet$ $p$ : "\texttt{x = alloca ty, i$n$ $t$}" 其中 $\texttt{x}\in\VP$ 且 $t\in\VP\cup\mathbb{Z}$; \\
~$\bullet$ $\not\vDash \lb a\rb \Dra (LV(t) > 0)$。 \\
\hline
\end{tabularx}
\end{table}

\begin{table}[htbp]
\caption{符号执行规则:\texttt{alloca} 指令(分配成功)}
\label{tab:rule-alloca}
\begin{tabularx}{\textwidth}{|X|}
\hline
\textbf{\texttt{alloca}} \emph{成功} \\
{\centering $
\inferrule
   {\lb p, LV, LAL, KB, AL, PT\rb}
   {\lb p^+, LV[\texttt{x} := v_1], LAL\cup\{\interp{v_1,v_2}\}, KB\cup\{v_2= v_1 + size(\ty)\cdot LV(t)-1\}, AL, PT\rb} 
$ \\}
\textbf{如果满足以下条件} \\
~$\bullet$ $p$ : "\texttt{x = alloca ty, i$n$ $t$}" 其中 $\texttt{x}\in\VP$ 且 $t\in\VP\cup\mathbb{Z}$; \\
~$\bullet$ $\vDash \lb a\rb \Dra (LV(t) > 0)$; \\
~$\bullet$ $v_1,v_2\in\Vsym$ 是新变量。 \\
\hline
\end{tabularx}
\end{table}

表~\ref{tab:rule-alloca-err} 和~\ref{tab:rule-alloca} 是 \verb|alloca| 指令的符号执行规则。分配内存成功的前提是,当前状态信息足以推断出分配的内存块大小为正数,否则则产生错误状态 $ERR$(如表~\ref{tab:rule-alloca-err})。当分配内存成功时,新状态的主要变化是其局部内存分配表 $LAL$ 中增加了一块新的内存区域 $\interp{v_1,v_2}$。符号变量 $v_1$ 与 $v_2$ 的关系在性质集合 $KB$ 中体现。


\begin{table}[htbp]
\caption{符号执行规则:\texttt{call} 指令}
\label{tab:rule-call}
\begin{tabularx}{\textwidth}{|X|}
\hline
\textbf{\texttt{call}} \\
{\centering $
\inferrule
   {\lb p, LV, LAL, KB, AL, PT\rb}
   {\lb p^+, LV[\texttt{x} := w], LAL, KB, AL, PT\rb} 
$ \\}
\textbf{如果满足以下条件} \\
~$\bullet$ $p$ : "\texttt{x = call ty $\ldots$}" 其中 $\texttt{x}\in\VP$;\\
~$\bullet$ $w\in\Vsym$ 是新变量。 \\
\hline
\end{tabularx}
\end{table}

\begin{table}[htbp]
\caption{符号执行规则:\texttt{ret} 指令}
\label{tab:rule-ret}
\begin{tabularx}{\textwidth}{|X|}
\hline
\textbf{\texttt{ret}} \\
{\centering $
\inferrule
   {\lb p, LV, LAL, KB, AL, PT\rb}
   {END}
$ \\}
\textbf{如果满足以下条件} \\
~$\bullet$ $p$ : "\texttt{ret ty $t$}" 其中 $t\in\VP\cup\mathbb{Z}$。\\
\hline
\end{tabularx}
\end{table}

表~\ref{tab:rule-call} 和~\ref{tab:rule-ret} 分别是 \verb|call| 指令和 \verb|ret| 指令的符号执行规则。由于本文采用单函数分析的方法,因此对 \verb|call| 指令与 \verb|ret| 指令的处理比文献~\inlinecite{DBLP:conf/cade/StroderGBFFHS14} 和~\inlinecite{DBLP:journals/jar/StroderGBFFHSA17} 精简。应用 \verb|call| 指令时,由于我们对所有函数逐一进行终止性分析,因此可以假设被 \verb|call| 调用的函数将返回某值。于是表~\ref{tab:rule-call} 的执行规则中,变量 \verb|x| 被赋予新的未知值。在程序中遇到 \verb|ret| 指令时,由于是单函数分析,如表~\ref{tab:rule-ret} 所示,下一状态为结束状态 $END$。

\begin{table}[htbp]
\caption{符号执行规则:抽象}
\label{tab:rule-abstract}
\begin{tabularx}{\textwidth}{|X|}
\hline
\emph{利用代换 $\mu$ 进行抽象} \\
{\centering $
\inferrule
   {\lb p, LV, LAL, KB, AL, PT\rb}
   {\lb p, LV', LAL', KB', AL', PT'\rb }
$ \\}
\textbf{如果满足以下条件} \\
~$\bullet$ $a$ 有一条求值类型的入边; \\
~$\bullet$ $LV$ 和 $LV'$ 的域相同,而且对所有 $\texttt{x} \in\VP$ 满足 $LV(\texttt{x}) = \mu(LV'(\texttt{x}))$; \\
~$\bullet$ $\vDash \lb a\rb \Dra \mu(KB')$;\\
~$\bullet$ 如果 $\interp{v_1,v_2}\in LAL'$,那么 $\interp{\mu(v_1),\mu(v_2)}\in LAL$; \\
~$\bullet$ 如果 $\interp{v_1,v_2}\in AL'$,那么 $\interp{\mu(v_1),\mu(v_2)}\in AL$; \\
~$\bullet$ 如果 $(v_1\pta_{\ty} v_2)\in PT'$,那么 $(\mu(v_1)\pta_{\ty}\mu(v_2))\in PT$。 \\
\hline
\end{tabularx}
\end{table}

最后是对抽象程序状态的抽象规则,如表~\ref{tab:rule-abstract} 所示。抽象后的新状态拥有同样的程序位置,新状态的信息可由原状态 $a$ 的一阶谓词集合推出。

有了符号执行规则,我们根据以下策略构造符号执行图~\cite{DBLP:conf/cade/StroderGBFFHS14,DBLP:journals/jar/StroderGBFFHSA17}:
\begin{itemize}
\item 
假设当前需要应用符号执行规则的状态顶点为 $b$,如果存在某个状态顶点 $a$ 满足:(i) 存在从 $a$ 到 $b$ 的一条路径;(ii) $a$ 和 $b$ 的程序位置 $p$ 相同;(iii) $a$ 和 $b$ 的 $LV$ 映射域相同;(iv) $b$ 存在一条求值类型的入边;(v) $a$ 不存在精化类型的入边。那么
\begin{itemize}
\item 如果状态 $a$ 是状态 $b$ 的抽象,那么构造一条抽象类型的边从 $b$ 指向 $a$。
\item 否则,移除 $a$ 的后继顶点,计算 $a$ 和 $b$ 的抽象 $c$ 并构造一条抽象类型的边从 $a$ 指向 $c$。如果 $a$ 已经存在来自某顶点 $q$ 的抽象类型入边,则将 $a$ 删除并构造一条抽象类型的边从 $q$ 指向 $c$。
\end{itemize}
\item
否则,对 $b$ 应用除抽象规则以外的符号执行规则,构造其后继顶点。
\end{itemize}

\subsection{整数变迁系统}

\begin{definition}[整数变迁系统]
整数变迁系统是一个三元组 $\lb S, C, \transto \rb$:
\begin{itemize}
\item $S$ 是一组状态集合 $\{s_1,\ldots,s_n\}$;
\item $C$ 是一组条件集合 $\{c_1,\ldots,c_m\}$;
\item ${\transto} \subseteq S \times C \times S$ 是一组变迁集合。
\end{itemize}
其中,条件 $c_i \subseteq QF\_IA(\cV\cup\cV')$ 是一组一阶公式,$\cV$ 是整数类型的变量集合,$\cV' = \{v' \mid v\in\cV \}$ 表示经过变迁后的变量值。
\end{definition}

整数变迁系统可以抽象地表示一个程序的状态空间和状态变迁。给定一个符号执行图,可以根据以下策略构造其对应的整数变迁系统~\cite{DBLP:conf/cade/StroderGBFFHS14,DBLP:journals/jar/StroderGBFFHSA17}:
\begin{enumerate} [(i)]
\item 符号执行图中每一个顶点 $a$ 都对应到整数变迁系统的一个状态 $s_a$;
\item 为符号执行图每一条从 $a$ 到 $b$ 的边构造一条对应的整数变迁系统的变迁 $\lb s_a, c, s_b\rb$:
\begin{itemize}
\item 如果 $\lb a, b\rb$ 不是一条抽象类型的边,则条件 $c = (\lb a\rb \cup \{v' = v | v\in\Vsym(a)\})$,其中 $\Vsym(a)$ 表示状态 $a$ 的所有符号化变量;
\item 如果 $\lb a, b\rb$ 是一条利用代换 $\mu$ 的抽象类型边,则条件 $c = (\lb a\rb \cup \{v'=\mu(v) \mid v\in\Vsym(b)\})$。
\end{itemize}
\end{enumerate}

\subsection{整数重写模型}
整数重写模型是规范化条件重写模型 $\RSE = \lb \cR,\cS,\cE \rb$ 的一种实例。当等价模型 $\cE$ 取整数代数运算符的性质集合(如加法交换律、乘法结合律等),化简模型 $\cS$ 取整数代数的计算规则时,规范化条件重写模型 $\RSE$ 实例化为整数重写模型,记作 $\cR_{\cI}$。

整数重写模型的重写规则通常具有如下形式:
\begin{eqnarray}
f(x_1,\ldots,x_n) & \ra & g(e_1,\ldots,e_m) \;\Dla\; \varphi \nonumber
\end{eqnarray}
其中,$e_1,\ldots,e_m$ 为整数代数表达式,$\varphi\in QF\_IA(\cV)$。

给定一个整数变迁系统,系统中的每一条变迁 $\lb s_i, c, s_j \rb$ 都可以用一条重写规则进行编码~\cite{DBLP:conf/cade/FalkeK09,DBLP:conf/rta/FalkeKS11}:
\begin{eqnarray}
s_i(x_1,\ldots,x_n) & \ra & s_j(e_1,\dots,e_n) \;\Dla\; \varphi_c \nonumber
\end{eqnarray}
其中,$x_1,\ldots,x_n$ 按 $\cV$ 中变量的固定序排列,且
\begin{itemize}
\item 如果 $c$ 中包含“赋值语句” $x_i' = p$,则 $e_i = p$;否则 $e_i = x_i$。
\item $\varphi_c$ 是 $c$ 中除去“赋值语句”后的所有一阶公式的合取。
\end{itemize}

\section{Ceagle-Termination}

\subsection{Ceagle-Termination 组成}

如图~\ref{f:Ceagle-Termination} 所示, Ceagle-Termination 主要由三部分构成:符号执行模块,转换模块和编码模块。前端接入第三方工具 Clang,将待验证的 C 程序编译成为 LLVM IR 程序。符号执行模块接受 LLVM IR 代码输入,生成对应的符号执行图。符号执行图经过转换模块后,输出整数变迁系统。该变迁系统经过编码模块的编码,生成适合后端重写模型终止性求解器的整数重写模型。后端接入第三方终止性求解器(目前支持开源求解器 \verb|KITTeL|),求解输出终止性验证结果。

\begin{figure}[ht]
\centering
\includegraphics[width=\textwidth]{Ceagle-Termination.jpg}
\caption{Ceagle-Termination 组成}
\label{f:Ceagle-Termination}
\end{figure}

Ceagle-Termination 的核心模块为符号执行模块,它涉及到对每一个抽象程序状态的符号执行、计算两个状态的抽象状态、以及对整个符号执行图进行构建。对状态的符号执行与抽象计算已经在小节\ref{ss:seg} 进行介绍,在此不再赘述。下面给出构建符号执行图的核心算法,见算法~\ref{alg:seg}。 

\begin{algorithm}[htbp]
  \KwIn{$G$:只包含初始状态顶点的符号执行图}
  \KwIn{$path$:存放顶点的栈结构,初始状态只包含初始顶点}
  \KwOut{$G$:构建完成的符号执行图}

  \While{ $path$ 非空 }{
    $v = peek(path)$\;
    \uIf{$v.visited == false$}{
      
      \If(\tcp*[h]{如果 $v$ 是结束状态}){$v == END$} {
        $v.visited = true$; \qquad $path.pop()$\;
        continue\;
      }

      %\tcp{decide whether we should generalize}
      在 $path$ 中寻找程序位置与 $v$ 相同的顶点,组成列表 $list_v$ \;
      %$list = path.findSamePC(v)$\;
      $should = false$ \tcp*{判断是否应该进行抽象}

      \ForEach{$u\in list_v$}{
        \If{ $u$ 和 $v$ 满足应该抽象的判断条件 }{
            %$u.domain() == v.domain()$ \\
            %$\land$ $v$ has an incoming evaluation edge \\
            %$\land$ $u$ has no incoming refinement edge}{
          $should = true$; \qquad $target = u$\;
          break\;
        }
      }

      %\tcp{Now we know what we should do}
      \uIf{should}{
        %\tcc{Do generalization}
        %perform Algorithm~\ref{a:generalization}\;
        应用算法~\ref{alg:abstraction} 进行抽象\;
      } \Else {
        %\tcc{Do evaluation}
        %perform Algorithm~\ref{a:evaluation}\;
        应用算法~\ref{alg:evaluation} 进行符号执行\;
      }
    } \Else(\tcp*[h]{ $v$ 已被访问过}) {
      在 $G$ 中寻找一个未被访问的 $v$ 的后继顶点 $w$\;
      \uIf{ $w$ 不存在} {
        $pop(path)$\tcp*{将 $v$ 从栈中取出}
      } \Else {
        $push(w,path)$\;
      }
    }
  }
\caption{构建符号执行图}
\label{alg:seg}
\end{algorithm}

\begin{algorithm}[htbp]
\KwIn{$G$:正在构建的符号执行图 }
\KwIn{$v$:正在访问的顶点 }
\KwIn{$target$:$v$ 的抽象目标顶点 } 
\KwIn{$path$:当前搜索路径}
\KwOut{$G$, $path$}
  \uIf{ 状态 $target$ 是 $v$ 的抽象 }{
    $v.visited = true$\;
    %$G.add\_edge(v,genTarget,generalization)$\;
    在 $G$ 中添加从 $v$ 到 $target$ 的一条抽象类型的边\;
    $pop(path)$\tcp*{将 $v$ 从栈中取出}
  } \Else(\tcp*[h]{ 需要计算 $v$ 和 $target$ 的抽象 }) { 
    %$path.popupto(genTarget)$\;
    对 $path$ 进行出栈操作,直到 $target$ 成为栈顶元素\;
    %$G.removeChild(genTarget)$\;
    在 $G$ 中删除 $target$ 的后继顶点\;
    %$c = merge(v, genTarget)$\;
    %$G.add\_vertex(c)$\;
    计算 $v$ 和 $target$ 的抽象状态 $c$,并将其加入 $G$\;

    \uIf{ 如果 $path$ 中只含 $target$ 一个元素 } {
      %$G.add\_edge(genTarget,c,generalization)$\;
      在 $G$ 中添加从 $target$ 到 $c$ 的一条抽象类型的边\;
      $push(c,path)$\;
    } \Else {
      $pop(path)$\tcp*{将 $target$ 从栈中取出}
      $p = peek(path)$\;
      \uIf{ 在 $G$ 中存在从 $p$ 到 $target$ 的抽象类型的边 }{
        从 $G$ 中删除顶点 $target$ 及其边\;
        在 $G$ 中添加从 $p$ 到 $c$ 的一条抽象类型的边\;
        $push(c,path)$\;
      } \Else {
        在 $G$ 中添加从 $target$ 到 $c$ 的一条抽象类型的边\;
        $push(target,path)$\;
        $push(c,path)$\;
      }
    }
  }
\caption{状态抽象}
\label{alg:abstraction}
\end{algorithm}

\begin{algorithm}[htbp]
\KwIn{$G$:正在构建的符号执行图 }
\KwIn{$v$:正在访问的顶点 }
\KwIn{$path$:当前搜索路径}
\KwOut{$G$, $path$}
  $v.visited = true$\;
  %$stateList = v.evaluate()$\;
  对状态 $v$ 应用除抽象外的符号执行规则,获得新状态列表 $list_{new}$\;
  \uIf(\tcp*[h]{没有对 $v$ 应用精化规则 }){$size(list_{new}) == 1$ } {
    $w = list_{new}[0]$\;
    在 $G$ 中添加顶点 $w$ 以及从 $v$ 到 $w$ 的一条求值类型的边\;
    $push(w,path)$\;
  } \Else(\tcp*[h]{ 精化规则被应用,产生2个新状态 }) {
    $w_1 = list_{new}[0]$\;
    $w_2 = list_{new}[1]$\;
    在 $G$ 中添加顶点 $w_1$ 以及从 $v$ 到 $w_1$ 的一条求值类型的边\;
    在 $G$ 中添加顶点 $w_2$ 以及从 $v$ 到 $w_2$ 的一条求值类型的边\;
    $push(w_1,path)$\;
  }
\caption{求值与精化}
\label{alg:evaluation}
\end{algorithm}


\subsection{Ceagle-Termination 性能评估}
\todo{需要等到工具原型开发完成}

\todo{与 AProVE 、\texttt{KITTeL}、Ctrl 进行比较}

\section{本章小结}

整数重写模型是规范化条件重写模型的一种特例。基于整数重写模型和符号执行图,我们开发了针对 C 语言程序终止性的自动验证工具 Ceagle-Termination。该工具接受 C 语言程序输入,并自动对其进行模型转换与终止性求解,不需要人工参与。

\todo{评价其表现}