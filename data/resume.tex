\begin{resume}

  \resumeitem{个人简历}

  1987年9月24日出生于广东省惠州市。

  2006年9月考入清华大学软件学院计算机软件专业,2010年7月本科毕业并获得工学学士学位。

  2010年 9 月免试进入清华大学计算机科学与技术系攻读工学学位至今。

  \researchitem{发表的学术论文} % 发表的和录用的合在一起

  % 1. 已经刊载的学术论文(本人是第一作者,或者导师为第一作者本人是第二作者)
  \begin{publications}
    \item Liu Jiaxiang, Dershowitz Nachum, Jouannaud Jean-Pierre. Confluence by critical pair
    analysis. In: Dowek G, (eds.). Rewriting and Typed Lambda Calculi - Joint 
    International Conference, RTA-TLCA 2014, Held as Part of the Vienna Summer 
    of Logic, VSL 2014, Vienna, Austria, July 14-17, 2014. Proceedings, volume 
    8560 of Lecture Notes in Computer Science. Springer, 2014. 287-302.
    \item Liu Jiaxiang, Jouannaud Jean-Pierre. Confluence: The unifying, expressive power of 
    locality. In: Iida S, Meseguer J, Ogata K, (eds.). Specification, Algebra, 
    and Software - Essays Dedicated to Kokichi Futatsugi,volume 8373 of Lecture 
    Notes in Computer Science. Springer, 2014. 337-358.
    \item Liu Jiaxiang, Jouannaud Jean-Pierre, Ogawa Mizuhito. Confluence of layered rewrite systems. 
    In: Kreutzer S, (eds.). 24th EACSL Annual Conference on Computer Science 
    Logic, CSL 2015, September 7-10, 2015, Berlin, Germany, volume 41 of 
    LIPIcs. Schloss Dagstuhl - Leibniz-Zentrum fuer Informatik, 2015. 423-440.
  \end{publications}

  % 2. 尚未刊载,但已经接到正式录用函的学术论文(本人为第一作者,或者
  %    导师为第一作者本人是第二作者)。
  \begin{publications}[before=\publicationskip,after=\publicationskip]
    \item Liu Jiaxiang, Zhou Min, Song Xiaoyu, Gu Ming, Sun Jiaguang. Formal modeling and verification of a 
    rate-monotonic scheduling implementation with Real-Time Maude. IEEE 
    Transactions on Industrial Electronics, 2016, PP(99):1-1. 
  \end{publications}

  % 3. 其他学术论文。可列出除上述两种情况以外的其他学术论文,但必须是
  %    已经刊载或者收到正式录用函的论文。
  \begin{publications}
    \item Jouannaud Jean-Pierre, Liu Jiaxiang.  From diagrammatic confluence to modularity.  
    Theoretical ComputerScience, 2012, 464:20-34.
  \end{publications}

  \researchitem{研究成果} % 有就写,没有就删除
  \begin{achievements}
    \item 任天令, 杨轶, 朱一平, 等. 硅基铁电微声学传感器畴极化区域控是一是事实上是哈是是是是是是是
      方法: 中国, CN1602118A. (中国专利公开号)
    \item Ren T L, Yang Y, Zhu Y P, et al. Piezoelectric micro acoustic sensor
      based on ferroelectric materials: USA, No.11/215, 102. (美国发明专利申请号)
  \end{achievements}

\end{resume}
