\chapter{定理~\ref{t:main} 的证明}
\label{app:proof}

\hide{
\setcounter{theorem}{0}
\renewcommand{\thetheorem}{\thesection.\arabic{theorem}}
}
%\newcommand{\lrps}[2]{\rightarrow^{#1}_{#2}}

本附录给出定理~\ref{t:main} 的详细证明。

为了证明定理~\ref{t:main} ,我们需要先介绍更多关于重写逻辑的背景知识~\cite{DBLP:journals/entcs/OlveczkyM07a}。如果一个项表达式 $t$ 不含变量,那么 $t$ 称作是\emph{封闭的}(ground)。给定原子命题的一个子集 $P\subseteq AP$ 和封闭的项表达式 $t$ 和 $t'$,如果 $t$ 满足的 $P$ 中的命题和 $t'$ 所满足的 $P$ 的命题完全一致,则记作 $t\simeq_P t'$ (如果 $P$ 可从上下文推断,则简记为 $t\simeq t'$)。

时间鲁棒性是我们希望一个实时重写理论所具有的一系列性质的集合。为了简化概念,我们在这里避免给出时间鲁棒性的准确定义,因为它对本附录的证明无关重要。我们给出以下引理用于证明时间鲁棒性:
\begin{lemma}[\inlinecite{DBLP:journals/entcs/OlveczkyM07a}]
\label{l:timerobustness}
给定一个面向对象风格的模型 $\mathcal{R^L}$,它只包含一条标准的单元计时规则,且满足时间域中只包含一个非正常的时间值为 \verb|INF|。如果 $\mathcal{R^L}$ 对所有封闭的项表达式 $t$、$r$ 和 $r'$ 满足以下条件,则 $\mathcal{R^L}$ 具有时间鲁棒性:
\begin{enumerate}[(i)]
\item 
对所有 $r\le$ \verb|mte(|$t$\verb|)| 满足
\verb|mte(delta(|$t$\verb|,|$r$\verb|))| $=$
\verb|mte(|$t$\verb|) monus |$r$,其中 \verb|monus| 是内置类型 \verb|Time| 的减法操作;

\item
\verb|delta(|$t$\verb|,|$0$\verb|)| $= t$;

\item
对所有 $r+r'\le$ \verb|mte(|$t$\verb|)| 满足
\verb|delta(delta(|$t$\verb|,|$r$\verb|),|$r'$\verb|)| $=$
\verb|delta(|$t$\verb|,|$r+r'$\verb|)|;

\item
对任意瞬时重写规则的左项表达式 $l$ 的所有封闭实例 $\sigma(l)$ 满足
\verb|mte(|$\sigma(l)$\verb|)|$= 0$。
\end{enumerate}
\end{lemma}

任意一步重写可以被分类,在此基础上,我们可以定义单元计时不变性。
\begin{definition}[\inlinecite{DBLP:journals/entcs/OlveczkyM07a}]
给定重写步骤 $t\lrps{r}{}t'$,假定它应用了单元计时规则,且其上标 $r$ 代表该重写的时间推移为 $r$:
\begin{itemize}
\item 
如果不存在时间值 $r'>r$ 使得对于某个 $t''$ 满足 $t\lrps{r'}{}t''$,则该重写步骤被称为一个\emph{极大的单元计时步骤}(maximal tick step),记作 $t\lrps{r}{max}t'$;

\item 
如果对任意时间值 $r'>0$,都存在某个 $t''$ 满足 $t\lrps{r'}{}t''$,则该重写步骤被称作一个\emph{$\infty$ 单元计时步骤}($\infty$ tick step),记作 $t\lrps{r}{\infty}t'$;

\item 
如果存在一个极大的单元计时步骤 $t\lrps{r'}{max}t''$ 满足 $r'>r$,则该重写步骤被称为一个\emph{非极大的单元计时步骤}(non-maximal tick step)。
\end{itemize}
\end{definition}

\begin{definition}[\inlinecite{DBLP:journals/entcs/OlveczkyM07a}]
一个时间鲁棒的模型 $\mathcal{R^L}$ 被称作根据一个命题集合 $P$ 保持单元计时不变,当且仅当,对任意非极大的单元计时步骤或 $\infty$ 单元计时步骤 $t\lrps{r}{}t'$ 有 $t\simeq_P t'$ 成立。
\end{definition}

我们需要以下引理证明我们的模型根据我们定义的命题具有单元计时不变性:
\begin{lemma}[\inlinecite{DBLP:journals/entcs/OlveczkyM07a}]
\label{l:tickinv}
给定一个时间鲁棒的、面向对象风格的模型 $\mathcal{R^L}$,它只包含一条标准的单元计时规则,且满足时间域中只包含一个非正常的时间值为 \verb|INF|。 
如果对于所有满足 $r<$ \verb|mte(|$t$\verb|)| 的 $t$ 和 $r$,都有 \verb|{|$t$\verb|}| $\simeq_P$ \verb|{delta(|$t$\verb|,|$r$\verb|)}| 成立,那么 $\mathcal{R^L}$ 是根据原子命题集合 $P$ 保持单元计时不变的。
\end{lemma}

\newcommand{\mteTask}[2]{\texttt{mteTask(}#1\texttt{,}#2\texttt{)}}
\newcommand{\deltaTask}[3]{\texttt{deltaTask(}#1\texttt{,}#2\texttt{,}#3\texttt{)}}
\newcommand{\mteIS}[1]{\texttt{mteIS(}#1\texttt{)}}
\newcommand{\deltaIS}[2]{\texttt{deltaIS(}#1\texttt{,}#2\texttt{)}}
\newcommand{\IntSrc}[3]{\texttt{<}#1\texttt{:IntSrc|val:}#2\texttt{,cycle:}#3\texttt{>}}
\newcommand{\mteIr}[1]{\texttt{mteIr(}#1\texttt{)}}
\newcommand{\mteS}[1]{\texttt{mte(}#1\texttt{)}}
\newcommand{\deltaS}[2]{\texttt{delta(}#1\texttt{,}#2\texttt{)}}
 
我们先证明以下中间引理:
\begin{lemma}
\label{l:auxtask}
给定分别具有类型 \verb|MaybeNat| 和 \verb|TaskList| 的项表达式 $\mathit{ID}$ 和 $L$,对所有 $r\le\mteTask{\mathit{ID}}{L}$,有 $\mteTask{\mathit{ID}}{\deltaTask{\mathit{ID}}{L}{r}} = \mteTask{\mathit{ID}}{L}\; \verb|monus|\;r$ 成立。
\end{lemma}
\begin{proof}
如果 $\mathit{ID}=\verb|none|$,则此情况显而易见。根据定义,有
\begin{eqnarray*}
& & \mteTask{\texttt{none}}{\deltaTask{\texttt{none}}{L}{r}} \\
& = & \mteTask{\texttt{none}}{L} \\
& = & \verb|INF| \\
& = & \mteTask{\texttt{none}}{L}~\verb|monus|~r \;\;\;\mbox{。}
\end{eqnarray*}

否则,存在具有类型 \verb|Nat| 的项表达式 $N$,使 $\mathit{ID}= \verb|some|~N$。假设 $L$ 的第 $N$ 个任务的 \verb|cnt| 值为 \verb|[|$r_e$\verb|/|$C$\verb|]|。则根据定义,$\deltaTask{\mathit{ID}}{L}{r}=L'$ 成立,其中 $L'$ 的第 $N$ 个任务的 \verb|cnt| 值等于 \verb|[|$(r_e+r)$\verb|/|$C$\verb|]|。因此,
\begin{eqnarray*}
& & \mteTask{\mathit{ID}}{\deltaTask{\mathit{ID}}{L}{r}} \\  
& = & \mteTask{\mathit{ID}}{L'} \\
& = & C~\verb|monus|~(r_e+r) \\
& = & (C~\verb|monus|~r_e)~\verb|monus|~r \\
& = & \mteTask{\mathit{ID}}{L}~\verb|monus|~r\;\;\;\mbox{。}
\end{eqnarray*}

以上,引理得证。
\end{proof}

\begin{lemma}
\label{l:auxis}
给定具有类型 \verb|IntSrc| 的项表达式 $\mathit{ISRC}$,它表示我们的模型中中断源的一个可达状态。则对所有 $r\le\mteIS{\mathit{ISRC}}$,$\mteIS{\deltaIS{\mathit{ISRC}}{r}} = \mteIS{\mathit{ISRC}}\;\verb|monus|~r$ 成立。
\end{lemma}
\begin{proof}
任意可达状态 $\mathit{ISRC}$ 必然具有以下形式 $\IntSrc{O}{v}{T}$,其中 $v\le T$。因此,
\begin{eqnarray*}
&  & \mteIS{\deltaIS{\mathit{ISRC}}{r}} \\  
& = & \mteIS{\IntSrc{O}{(v~\texttt{monus}~r)}{T}} \\
& = & v~\texttt{monus}~r \\
& = & \mteIS{\mathit{ISRC}}~\texttt{monus}~r\;\;\;\mbox{。}
\end{eqnarray*}

以上,引理得证。
\end{proof}

\begin{lemma}
\label{l:auxhw}
给定具有类型 \verb|Hardware| 的项表达式 $\mathit{HW}$,则对所有项表达式 $r$,如果满足 $r\le\mteIr{\mathit{HW}}$,那么 $\mteIr{\mathit{HW}} = \mteIr{\mathit{HW}}\; \verb|monus|\; r$ 成立。
\end{lemma}
\begin{proof}
对项表达式 $\mathit{HW}$ 中是否存在被检测到的中断请求进行分情况讨论,则引理可证。
\end{proof}

定理~\ref{t:main} 的详细证明如下:

\begin{proof}[定理~\ref{t:main} 的证明]
我们先利用引理~\ref{l:timerobustness} 证明我们建立的模型满足时间鲁棒性。由于我们的模型选择了内置的连续时间域 \verb|POSRAT-TIME-DOMAIN-WITH-INF| 进行实例化,因此 \verb|INF| 是模型中唯一的非正常时间值。如第~\ref{ss:timedbehavior} 小节所述,在模型中我们只定义了一条标准单元计时规则。因此,根据引理~\ref{l:timerobustness},如果条件~(i--iv) 成立,那么我们的模型满足时间鲁棒性。

条件~(i)。我们需要证明,对所有 $r\le$ \verb|mte(|$s$\verb|)|,均有 \verb|mte(delta(|$s$\verb|,|$r$\verb|))| $=$ \verb|mte(|$s$\verb|)| \verb|monus| $r$ 成立,其中 $s$ 是具有类型 \verb|System| 的系统状态项表达式。假设 $s$ 具有形式 $(L~T~\mathit{STS}~\mathit{HW}~\mathit{ISRC})$,且 $\mathit{ID}=\verb|(|\mathit{HW}\verb|).getPc|$。在此我们详细讨论 $\mathit{ID}$\verb|::MaybeNat| 成立的情况,另一情况 $\mathit{ID}$\verb|::Oid| 的证明过程类似。根据 \verb|mte| 和 \verb|delta| 的定义,
\begin{eqnarray*}
&  & \mteS{\deltaS{s}{r}} \\  
& = & \mteS{\deltaTask{\mathit{ID}}{L}{r}~\;T~\mathit{STS}~\mathit{HW}~\deltaIS{\mathit{ISRC}}{r}} \\
& = & \verb|minimum(|\mteTask{\mathit{ID}}{\deltaTask{\mathit{ID}}{L}{r}}\verb|,| \\
&  & \verb|        |\mteIS{\deltaIS{\mathit{ISRC}}{r}}\verb|,| 
\;\;\mteIr{\mathit{HW}}\verb|)|\;\;\;\mbox{。}
\end{eqnarray*}
根据引理~\ref{l:auxtask}、引理~\ref{l:auxis} 和引理~\ref{l:auxhw},以上表达式可以进一步化简:
\begin{eqnarray*}
&  & \mteS{\deltaS{s}{r}} \\  
& = & \verb|minimum(|\mteTask{\mathit{ID}}{L}~\verb|monus|~r~\verb|,| \\
&  & \verb|        |\mteIS{\mathit{ISRC}}~\verb|monus|~r~\verb|,| \\
&  & \verb|        |\mteIr{\mathit{HW}}~\verb|monus|~r~\verb|)| \\
& = & \verb|minimum(|\mteTask{\mathit{ID}}{L}\verb|,| \\
&  & \verb|        |\mteIS{\mathit{ISRC}}\verb|,| \\
&  & \verb|        |\mteIr{\mathit{HW}}\verb|)|~\verb|monus|~r \\
& = & \mteS{s}~\verb|monus|~r\;\;\;\mbox{。}
\end{eqnarray*}

条件~(ii)。由于对所有 $r$ 都有 $r+0=r$ 和 $r~\verb|monus|~0=r$ 成立,因此条件~(ii) 成立。

条件~(iii)。我们需要证明 $\deltaS{\deltaS{s}{r}}{r'} = \deltaS{s}{r+r'}$ 对所有 $r+r'\le\mteS{s}$ 成立,其中 $s$ 是具有类型 \verb|System| 的系统状态项表达式。采用与条件~(i) 相同的符号,我们给出 $\mathit{ID}$\verb|::MaybeNat| 成立时的详细证明。根据定义,待证等式的左边
\begin{eqnarray*}
&  & \deltaS{\deltaS{s}{r}}{r'} \\  
& = & \verb|(|\deltaTask{\mathit{ID}}{\deltaTask{\mathit{ID}}{L}{r}}{r'} \\
&  & \verb| |~T~\mathit{STS}~\mathit{HW}~\deltaIS{\deltaIS{\mathit{ISRC}}{r}}{r'}\verb|)|~\mbox{,}
\end{eqnarray*}
而待证等式的右边
\begin{eqnarray*}
&  & \deltaS{s}{r+r'} \\  
& = & \verb|(|\deltaTask{\mathit{ID}}{L}{r+r'} ~T~\mathit{STS}~\mathit{HW}~\deltaIS{\mathit{ISRC}}{r+r'}\verb|)|~\mbox{。}
\end{eqnarray*}
由于 $+$ 具有结合律,因此 
\begin{eqnarray*}
\deltaTask{\mathit{ID}}{\deltaTask{\mathit{ID}}{L}{r}}{r'}
& = & \deltaTask{\mathit{ID}}{L}{r+r'}
\end{eqnarray*}
成立。又因为 $(v\; \verb|monus|\; r)\; \verb|monus|\; r' = v\;
\verb|monus|\; (r+r')$ 对所有 \verb|Time| 类型的项表达式 $v$ 成立,于是
\begin{eqnarray*}
\deltaIS{\deltaIS{\mathit{ISRC}}{r}}{r'} & = & \deltaIS{\mathit{ISRC}}{r+r'}
\end{eqnarray*}
成立。综上,条件~(iii) 成立。

条件~(iv)。我们证明任意瞬时规则的左项实例的 \verb|mte| 值等于 $0$。比如考虑规则 \verb|interrupt-request|,根据其条件约束 \verb|(|$\mathit{ISRC}$\verb|).timeout|,可得知 $\mathit{ISRC}$ 的 \verb|val| 值等于 $0$。所以,
\begin{eqnarray*}
&  & \mteS{L~T~\mathit{STS}~\mathit{HW}~\mathit{ISRC}} \\  
& = & \verb|minimum(|\mteTask{\mathit{ID}}{L}\verb|,| \mteIS{\mathit{ISRC}}\verb|,| \mteIr{\mathit{HW}}\verb|)| \\
& = & \verb|minimum(|\mteTask{\mathit{ID}}{L}\verb|,|~0~\verb|,| \mteIr{\mathit{HW}}\verb|)| \\
& = & 0\;\;\;\mbox{。}
\end{eqnarray*}
以此类推,其它规则也可根据其条件约束得到证明。综上所述,根据引理~\ref{l:timerobustness} 可证我们的模型满足时间鲁棒性。

最后,我们证明用于分析模型的命题(即 \verb|taskTimeout| 和 \verb|correct|)满足单元计时不变性。根据引理~\ref{l:tickinv},我们必须证明 
$\verb|{|s\verb|}|\simeq_P\verb|{|\deltaS{s}{r}\verb|}|$ 对所有 \verb|System| 类型的项表达式 $s$ 和 $r<\mteS{s}$ 成立。也就是说,对系统状态项表达式应用单元计时规则使得时间往前推移 $r$ 个时间单位,并不会对任意命题的真值产生影响。假设 $s$ 的形式为 $(L~T~\mathit{STS}~\mathit{HW}~\mathit{ISRC})$ 且 $\verb|(|\mathit{HW}\verb|).getPc|=\mathit{ID}$。
\begin{itemize}
\item 
命题 \verb|taskTimeout| 成立当且仅当 $L$ 中包含 \verb|error|。由于 \verb|delta| 并不会在 $L$ 中产生或减少 \verb|error|,因此命题 \verb|taskTimeout| 针对状态 $s$ 成立,当且仅当, \verb|taskTimeout| 针对状态 $\deltaS{s}{r}$ 成立,其中 $r<\mteS{s}$。这意味着我们的模型根据 \verb|taskTimeout| 保持单元计时不变性。
\item 
命题 \verb|correct| 的真值取决于 $\mathit{ID}$ 以及 $L$ 中所有任务的状态。与上述情况类似,\verb|delta| 无法改变 $\mathit{ID}$ 或 $L$ 中任意任务的状态,所以命题 \verb|correct| 在状态 $s$ 成立, 当且仅当, \verb|correct| 在状态 $\deltaS{s}{r}$ 也成立,其中 $r<\mteS{s}$。模型根据命题 \verb|correct| 的单元计时不变性得证。
\end{itemize}

综上所述,根据定理~\ref{t:completeness},我们应用非时控模型检测对可调度性及系统正确性进行验证的方法具有完备性。
\end{proof}
