\chapter{结束语}
\label{cha:conclusion}

\section{工作总结}

针对现代嵌入式系统结构复杂、行为复杂所引发的形式化建模与验证问题,本文基于重写理论,完成了以下工作:

\begin{enumerate}
\item 
针对嵌入式系统中硬件的并发行为与软件的顺序行为并存的异构性特征进行研究,提出了规范化条件重写模型。将硬件的并发行为抽象成不确定性行为,将软件的顺序行为抽象成确定性行为,基于模重写模型、条件重写模型以及 Nipkow 在高阶重写中使用的 $\beta\eta$ 规范化过程,本文提出的规范化条件重写模型,利用等式规则描述系统的结构特征及状态等价关系,利用条件约束描述系统的控制流特征,利用化简规则描述系统的确定性行为,利用重写规则描述系统的不确定性行为。本文对重写模型的规则应用策略进行扩展,定义了在多种规则共存的情况下,确定性行为与不确定性行为的协作方式,且保证了重写规则触发时条件约束的可判定性。规范化条件重写模型具有严格的形式化语法及语义定义。该模型是面向建模与验证的重写模型扩展,相比其它重写模型扩展,它的表达能力得到了提升。

\item
基于规范化条件重写模型,以嵌入式系统建模方法论为切入点就重写模型易用性低的问题进行研究。本文提出对嵌入式系统的结构层次性、行为异构性、结构动态性和实时性等特征的具体建模方法,对系统建模过程进行指导。基于语义映射的方式,将该建模方法在工具 Maude 中进行实现。通过对两个真实嵌入式系统的建模验证过程予以应用,验证了该建模方法在嵌入式系统中实际应用的可行性。其中,机车优化控制系统在本文的建模与验证过程中,成功定位了测试人员进行仿真测试未能发现的系统缺陷;该系统目前运行稳定,满足了铁路节能驾驶使用要求,并在沈阳铁路局通过了实车运用考核。速率单调调度系统的可调度性和正确性通过模型检测方法得到了验证,且本文证明了该结果具有可靠性和完备性;该调度系统目前在某工业级航天控制器中在线运行。

\item 
基于整数重写模型这一规范化条件重写模型实例,针对重写模型易用性低的问题,以嵌入式系统软件的自动建模为切入点进行研究。本文设计、开发了针对 C 语言程序终止性的自动验证工具 \CTerm。为增加工具的可扩展性,\CTerm 采用 LLVM IR 作为程序的中间表示语言。参考已有工具,\CTerm 采用符号执行技术生成符号执行图作为程序行为的中间模型。它的后端可以接入多种重写模型终止性求解器,进一步提高了它的可扩展性。 \CTerm 接受 C 语言程序输入,自动生成终止性验证结果,有效降低了形式化验证方法的使用成本,且为软件的完全正确性提供了必要的支持。 
\end{enumerate}

\section{研究展望}

在本文的工作基础上,为了进一步推动规范化条件重写模型在嵌入式系统建模与验证工作中的应用,拟在以下方面开展更多研究工作:

\begin{enumerate}
\item 
对条件模重写模型的终止性与合流性开展研究。在规范化条件重写模型 $\lb \cR,\cS,\cE\rb$ 的定义中,$\cS$ 和 $\cE$ 被要求使得条件模重写模型 $\SE$ 满足终止性和合流性。从建模的角度看,根据第~\ref{s:modeling} 小节提出的建模方法,$\cS$ 用于建模系统的确定性行为(如软件的局部顺序行为),$\cE$ 用于描述项表达式的结构信息,因此 $\SE$ 的终止性和合流性可以由建模的方法得到保证。然而建模过程是由人主导的,人总是容易出错,这正是为什么需要对系统进行验证的原因。从理论和工具的角度对 $\SE$ 的终止性和合流性进行检查,不但可以保证模型的一致性(consistency),还可以发现建模过程引入的错误。虽然目前存在对模重写模型的终止性和合流性的研究\cite{DBLP:conf/cade/JouannaudM84,DBLP:journals/tcs/JouannaudM92,DBLP:journals/ijsi/JouannaudT08,DBLP:conf/rta/Jouannaud06,DBLP:journals/tcs/JouannaudL12,DBLP:journals/siamcomp/JouannaudK86},但目前的结果对等价模型 $\cE$ 具有较严格的约束。因此,要推动规范化条件重写模型在建模过程中的应用,需要对模重写模型的终止性和合流性进行更一般化的理论研究,以及相应检查工具的开发。
\item
进一步提高和完善基于规范化条件重写模型的建模方法。现代嵌入式系统种类繁多、行为复杂,本文提出的基于规范化条件重写模型的建模方法,需要在更多实际案例上进行应用,才能发现它在应对不同系统时产生的问题,对其进一步提高和完善。对该建模方法的另一种可行的完善方式,是针对不同的特定嵌入式系统,如中断系统、调度系统等,设计领域特定语言(Domain Specific Language,DSL)来对特定系统进行建模,并将 DSL 模型翻译成对应的规范化条件重写模型。
\item 
完善规范化条件重写模型的支持工具集。如第~\ref{ss:method-impl} 小节所述,工具 Maude 不能完全支持规范化条件重写模型的建模,其根本原因在于 Maude 的理论模型——重写逻辑的表达能力不如规范化条件重写模型。因此,对 Maude 的底层核心模块进行扩展,或开发专门针对规范化条件重写模型的建模验证工具集,可以更充分地发挥规范化条件重写模型的作用。
\item 
提高和完善 \CTerm 工具。一方面,由于 \CTerm 采用符号执行技术对模型进行构建,这导致在大规模程序上进行应用时要耗费大量的计算资源。因此,为了降低模型构建的资源占用问题,也为了减小模型规模,可以考虑使用诸如切片~\cite{DBLP:journals/tse/Weiser84} 等程序分析变形技术,在构建模型前对程序中与终止性无关的代码进行削减。另一方面,在构造程序的符号执行图时,抽象规则会导致状态的信息丢失。而在丢失的状态信息中,可能存在某些对终止性判定有用的信息,比如循环不变式~\cite{DBLP:journals/cacm/Hoare69}。设计新的抽象规则,使更多的有用信息得到保留,同时尽量减少不必要的信息,是一个值得研究的方向。再者,虽然目前的内存模型以及对函数调用的执行规则可以保证符号执行图的构造过程在递归函数存在时得以终止,但这并没有完全解决对递归函数的建模问题。如何对递归函数的语义进行精确建模,这是一个从理论和应用层面都值得研究的问题。最后,将不同的终止性验证技术、非终止性验证技术与 \CTerm 进行融合,是完善 \CTerm 工具的另一个方向。
\end{enumerate}